%\documentclass[twocolumn,10pt]{article}
%\usepackage{hyperref}

\documentclass[applications]{gen-bioinformatics}

\makeatletter
\renewcommand\boldmath{\@nomath\boldmath\mathversioo
n{bold}}
\makeatother

\definecolor{BiocBlue}{RGB}{24,129,194}
\usepackage{amssymb,amsfonts,url,times}
\usepackage[linkcolor=BiocBlue,pdfborder={0 0 0},urlcolor=BiocBlue]{hyperref}
\usepackage{graphics}
\usepackage{amsmath}
\usepackage{fontawesome}
\usepackage{dcolumn}
\newcolumntype{.}{D{.}{.}{-1}}
%\usepackage{hlight}

\urlstyle{rm}
\def\email#1{#1}



\newcommand{\Rfunction}[1]{{\texttt{#1}}}
\newcommand{\Robject}[1]{{\texttt{#1}}}
\newcommand{\Rpackage}[1]{{\textit{#1}}}
\newcommand{\BiocpackageFirstBAD}[1]{{\emph{\href{https://bioconductor.org/packages/3.8/#1}{#1\textsubscript{\faExternalLink}}}}} 
\newcommand{\Biocpackage}[1]{{\textit{#1}}}
\newcommand{\BiocpackageFirst}[1]{{\textit{#1}}}
\newcommand{\CRANpackage}[1]{{\emph{\href{https://cran.r-project.org/web/packages/#1/index.html}{#1}}}}
\newcommand{\CRANpackageFirst}[1]{{\emph{\href{https://cran.r-project.org/web/packages/#1/index.html}{#1}}}}
\newcommand{\CRANpackageFirstBAD}[1]{{\emph{\href{https://cran.r-project.org/web/packages/#1/index.html}{#1\textsubscript{\faExternalLink}}}}}
\newcommand{\Rmethod}[1]{{\texttt{#1}}}
\newcommand{\Rfunarg}[1]{{\texttt{#1}}}
\newcommand{\Rclass}[1]{{\texttt{#1}}}
\providecommand{\OO}[1]{\operatorname{O}\left(#1\right)}
 

\author[1]{\pfnm{Shweta}
  \pinit{}
  \psnm{Gopaulakrishnan}}

\author[1]{\pfnm{Samuela}
  \pinit{}
  \psnm{Pollack}}

\author[1]{\pfnm{Benjamin}
  \pinit{}
  \psnm{Stubbs}}

\author[2]{\pfnm{Herv\'e}
  \pinit{}
  \psnm{Pag\`es}}

\author[3]{\pfnm{John}
  \pinit{}
  \psnm{Readey}}

\author[4]{\pfnm{Sean}
  \pinit{}
  \psnm{Davis}}

\author[5]{\pfnm{Levi}
  \pinit{}
  \psnm{Waldron}}

\author[6]{\pfnm{Martin}
  \pinit{T}
  \psnm{Morgan}}

\author[1]{\pfnm{Vincent}
  \pinit{J}
  \psnm{Carey}}

\address[1]{\porgdiv{Channing Division of Network Medicine}
  \porgname{Brigham and Women's Hospital}
  \pstreet{181 Longwood Avenue }
  \pcity{Boston}
  \postcode{02115}
  \pcnty{USA}}

\address[2]{\porgdiv{Systems Bioinformatics}
  \porgname{Fred Hutchinson Cancer Research Center}
  \pstreet{Fairview Avenue }
  \pcity{Seattle}
%  \postcode{}
  \pcnty{USA}}

\address[3]{
  \porgname{HDF Group}
%  \pstreet{}
  \pcity{Seattle}
%  \postcode{}
  \pcnty{USA}}

\address[4]{\porgdiv{Center for Cancer Research}
  \porgname{NCI}
  \pcity{Bethesda}
  \pcnty{USA}}


\address[5]{
  \porgdiv{Institute for Implementation Science in Population Health}
  \porgname{CUNY School of Public Health}
  \pcity{NY}
  \pcnty{USA}}

\address[6]{
  \porgname{RPCI}
%  \pstreet{}
  \pcity{Buffalo}
%  \postcode{}
  \pcnty{USA}}

 

\begin{document}


\title{restfulSE: a semantically rich interface for cloud-scale genomics
with Bioconductor}
\maketitle

\begin{abstract}
\begin{subabstract}[Summary]
Bioconductor's \texttt{SummarizedExperiment} class unites numerical
assay quantifications with sample- and experiment-level metadata.  
It is the standard Bioconductor class for assays that
produce matrix-like data, used by over 200 packages.
We describe the \Biocpackage{restfulSE} package, a deployment of 
this data model that supports
remote storage.
We illustrate use of
\texttt{SummarizedExperiment} with remote HDF5 and Google
BigQuery back ends, with two applications in cancer genomics.
\end{subabstract}
\begin{subabstract}[Availability] Package \BiocpackageFirst{restfulSE} of Bioconductor
 (\url {www.bioconductor.org}).  Artistic-2.0 licensing.  Additional
open software dependencies are enumerated in the supplement.
\end{subabstract}
\begin{subabstract}[Contact]reshg@channing.harvard.edu
\end{subabstract}
%\begin{subabstract}[keywords] REST API, HDF5, Bioconductor
\end{abstract}
\section*{Introduction}

The analysis of multiomic archives like The Cancer Genome Atlas (TCGA, \url{https://cancergenome.nih.gov/})
and single-cell transcriptomic experiments such as the 10x 1.3 million
mouse neuron dataset (\url{https://support.10xgenomics.com/single-cell-gene-expression/datasets}) typically begin with downloads of large files and
conversion of file contents into formats based on local preferences.
In this paper we consider how targeted queries of large remote genomic
data resources can be conducted using methods available for
Bioconductor's \texttt{SummarizedExperiment} class.  
For large data archives that have been centralized
in cloud storage, use of this approach can help diminish effort
required to manage local storage,
and can facilitate interactive analysis of data
subsets in familiar programming idioms, 
without downloading entire datasets. 
%Clients for
%HDF5 (\url{https://www.hdfgroup.org/}) or BigQuery (\url{https://cloud.google.com/bigquery/}) are available in numerous languages; our
%Bioconductor interface permits access to remote archives of genomic
%data with familiar and semantically meaningful programmatic idioms,
%while abstracting the remote interface from end users
%and developers.

\section*{Description}

%\subsection*{The \texttt{SummarizedExperiment} class and related methods}

%Let $Q$ denote a matrix of quantifications arising from a genome
%scale assay with $G$ assay features measured on $N$ experimental
%samples.  The elements of $Q$ are the numbers $q_{ij}, i = 1, \ldots, G,
%j = 1, \ldots, N$.  Bioconductor's \texttt{SummarizedExperiment} structure

Bioconductor's \texttt{SummarizedExperiment} structure
manages feature quantifications
with associated metadata about assay features
and samples; see the Supplementary Material for details.
With \Biocpackage{restfulSE}, the quantifications reside in
remote storage, but programmatic access proceeds via familiar
idioms and enhances reliability through tight binding
of metadata to assay data.  We provide examples of assays
stored in Google BigQuery and HDF Scalable Data Service (HSDS).

%In the 10x mouse neuron dataset, $G=27998$ and $N=1.3$ million.
%When these quantifications are managed in a Bioconductor \texttt{SummarizedExperiment X}, the matrix $Q$ is programmatically bound to a $G \times F$
%table of feature-level metadata (e.g., gene or transcript names and
%characteristics) accessible by the \texttt{rowData} method, and to an $N \times R$ table of sample-level metadata accessible by \texttt{colData} \citep{Huber2015}. 
%The standard subsetting idiom \texttt{X[G,S]} expresses filtering of 
%the all the information in $Q$ and the associated metadata
%to features \texttt{G} and samples \texttt{S}.  A \texttt{GRanges} 
%instance \citep{Lawrence2013} defining genomic coordinates for features may be bound to \texttt{X},
%facilitating queries defined by genomic location (using, for example, \texttt{subsetByOverlaps}) to isolate features
%coincident with or near the elements of a set of query genomic ranges (eg., binding peaks).  This outline of genomic data representation
%and analysis is characteristic of Bioconductor.

%\textbf{Google BigQuery.} The Institute for Systems Biology Cancer
%Genomics Cloud project (ISB-CGC) \citep{ISBCGC} uses 
%Google BigQuery to provide access to
%various public cancer genomics resources including
%TCGA and the PanCancer Atlas \citep{Hoadley2018}.
%The \verb+pancan_SE+
%function of \Biocpackage{restfulSE} constructs queries that derive
%\texttt{SummarizedExperiment} instances using quantifications and annotations
%for PanCancer atlas experiments
%managed in BigQuery tables.  
%
%\noindent
%\textbf{HDF Cloud (Scalable Data Service).}
%An AWS S3-based distributed data object model for
%HDF5 datasets, including a
%RESTful API to structure, populate, and query HDF5 archives, 
%has been implemented by the HDF Group. 

%\noindent
%\textbf{DelayedArray.}
%The \Biocpackage{restfulSE} package provides interfaces to 
%BigQuery and HDF Cloud so that 
%the numerical content housed in these services
%satisfies the API of the Bioconductor \BiocpackageFirst{DelayedArray} package \citep{Pages2018}.  
%%Any \verb+DelayedArray+ instance can serve as the \verb+assay+
%component of a \texttt{SummarizedExperiment} instance.  Thus the
%capacities of \verb+SummarizedExperiment+ to bind semantically
%rich metadata to genome-scale assays are extended implicitly to
%data resources for which no standards exist for
%associating substantive metadata.  
%In conjunction with the \BiocpackageFirst{rhdf5client} \citep{rclient} and \CRANpackageFirst{bigrquery} \citep{bigr} packages,
%\Biocpackage{restfulSE} translates filtering and selection operations
%which are readily defined using \verb+rowData+, \verb+rowRanges+,
%and \verb+colData+ into formal queries resolvable by the HDF5 and
%BigQuery services.  Numerical results are transmitted from
%server to client only when needed.

\section*{Results}

%The RESTful \texttt{SummarizedExperiment} representation
%allows complicated research queries to be obtained in a concise,
%fast, convenient and robust fashion, as illustrated by
%the following examples.

\textbf{BigQuery back end.} Figure \ref{bladcan} illustrates the 
use of the \Biocpackage{restfulSE} protocol
with the ISB-CGC BigQuery back end.  
The \texttt{buildPancanSE} function
of the \BiocpackageFirst{BiocOncoTK} package produces
a RESTful SummarizedExperiment instance based on
user selections of tumor and assay type.  By default,
the Infinium 450K methylation assay results for bladder tumors
are retrieved.  The \texttt{buildPancanSE} function
will trigger an authentication event.  After successful
authentication, sample-level (clinical) data are retrieved
along with assay feature names, to establish a shell
of metadata regarding the selected components of the
pancancer atlas.

\begin{figure}
{\small
\begin{verbatim}
library(BiocOncoTK)
library(SummarizedExperiment)
bq = pancan_BQ(quiet=TRUE)  # (a)
blcaMeth = buildPancanSE(bq) # (b) need auth.
blcaMeth
# class: RangedSummarizedExperiment 
# dim: 396065 409 
# metadata(3): acronym assay sampType
# assays(1): assay
# rownames(396065): cg00000029 cg00000165 ...
#   rs966367 rs9839873
# rowData names(3): gene_id gene_name
#   gene_biotype
# colnames(409): TCGA-FD-A5BV TCGA-FD-A3B3 ...
#   TCGA-UY-A9PH TCGA-UY-A78L
# colData names(21): bcr_patient_uuid
#   bcr_patient_barcode ... race
#   neoplasm_histologic_grade
\end{verbatim}
}
\caption{
Code and outputs for a) connecting with 
BigQuery resource for pancancer-atlas, b)
constructing a RESTful SummarizedExperiment
for the Illumina 450K Methylation assay obtained
on bladder tumors.
%, c) interrogating metadata to
%determine indices of CpG features nearest to 
%TSS of genes GP5 and ZSCAN12, d) acquiring a
%preview of quantifications of the associated probes,
%and e) tabulating histologic grade for the
%bladder cancer samples.
}
\label{bladcan}
\end{figure}



\noindent
\textbf{HDF Cloud back end.}  Figure \ref{hdffig}
demonstrates use of a RESTful \texttt{SummarizedExperiment},
with assay data \texttt{/shared/bioconductor/darmgcls.h5}
at \texttt{hsdshdflab.hdfgroup.org}.  Briefly, as a
prelude to single-cell RNA-sequencing of glioblastoma (GBM)
tumors from four patients,
\cite{Darmanis2017} used immunopanning to increase the
proportion of non-neoplastic cells that constitute
the "migrating front" of progression of glioblastoma.
Antibody to CD45 was used to capture microglial cells.
Figure \ref{hdffig} provides code to compare
the distribution of CD45 expression among the
classes of
cells as labeled in the metadata of GSE84465,
the NCBI GEO archive from which the quantifications
were derived.  
%In this example, data on one
%gene from all cells
%is retrieved when the statement defining vector \texttt{vals}
%is executed.  The display can be recapitulated for
%other genes by substituting different symbols in
%the statement computing \texttt{ind}.
%The \verb+DelayedArray+ framework leveraged here
%enables basic computations of this kind without loading the
%entire matrix into memory.



\begin{figure}
\small{
\begin{verbatim}
library(BiocOncoTK)
darm = BiocOncoTK::darmGBMcls
darm
# class: RangedSummarizedExperiment 
# dim: 65218 3584 
# metadata(1): source
# assays(1): count_lstpm
# rownames(65218): ENSG00000000003.14
#   ENSG00000000005.5 ... ERCC-00170
#   ERCC-00171
# rowData names(3): gene genome
#   symbol
# colnames(3584): GSM2243439
#   GSM2243440 ... GSM2247076
#   GSM2247077
# colData names(59): title
#   geo_accession ...
#   tsne.cluster.ch1 well.ch1

sapply(split(as.numeric(
   assay(darm["ENSG00000081237.18",])), 
   darm$characteristics_ch1.8), median)
#
# selection: Astrocytes(HEPACAM) 
#                       0.000000 
#    selection: Endothelial(BSC) 
#                       0.000000 
#     selection: Microglia(CD45) 
#                      35.113292 
#       selection: Neurons(Thy1) 
#                       0.000000 
#selection: Oligodendrocytes(GC) 
#                       6.764196 
#            selection: Unpanned 
#                       0.000000 
\end{verbatim}
}
\caption{Code and cell-type specific
medians of CD45 expression across immunopanned
cells reported in \cite{Darmanis2017}.
\texttt{BiocOncoTK::darmGBMcls} includes
a reference to the HDF Scalable Data Service
(HDF Cloud) representation of
this glioblastoma single-cell RNA-seq
as requantified in the CONQUER project \citep{Soneson2018}.}
\label{hdffig}
\end{figure}



\noindent
%\textbf{HDF Cloud back end.}  Figure \ref{hdffig}
%demonstrates construction of a RESTful \texttt{SummarizedExperiment}
%for expression data on 1.3 million neurons.
%The syntax \verb+tenx[,s]+ generates a delayed selection for
%all genes with samples 
%identified in vector \verb+s+.  Invoking the \verb+assay()+
%method triggers RESTful queries to the object store at \verb+URL_hsds()+.
%These can be serviced in parallel, with
%throughput dependent on the configuration
%of the HSDS server and its load at time of request.
%Replies can be requested in JSON or binary formats;
%\Rpackage{rhdf5client} utilities handle parsing the
%responses into R matrix structures.
%In this example, data for only four of 1.3 million neurons
%are downloaded and used to compute sums of RNA-seq read
%counts.  The \verb+DelayedArray+ framework leveraged here
%enables basic computations of this kind without loading the
%entire matrix into memory.



%\section*{Performance}
%
%We focus on pursuit of reliability,
%expressivity, and scalability using \Biocpackage{restfulSE}.  
%\textit{Reliability:} 
%The \Biocpackage{restfulSE}, \Biocpackage{rhdf5client} \citep{rclient},
%and \Biocpackage{BiocOncoTK} \citep{bionc} packages are accompanied by detailed unit
%tests that compare retrievals to known values.  In the
%case of BigQuery table queries, the test
%suite composes random queries 
%in both BigQuery SQL and in the \texttt{SummarizedExperiment} 
%idiom.  Results
%are checked for elementwise equality.  \textit{Expressivity:} The code
%segments in Figures \ref{pancanPanel} and \ref{hdffig} are
%complex but easy to break down.  The joining and
%reshaping of pancan-atlas tables in BigQuery corresponding
%to the code in Figure \ref{pancanPanel}
%can be checked through the query history in the BigQuery
%interface.  The acquisition of expression values required
%five nested SELECT statements; the query for assay quantifications
%was 6000 characters in length.
%The R code is 223 characters including comments.
%\textit{Scalability.}  BigQuery is intrinsically auto-scaling,
%but charges accrue with the amount of data scanned, 
%so query design can have effects on throughput
%and cost.  We rely on the \CRANpackage{bigrquery} \citep{bigr} and \CRANpackageFirst{dbplyr} \citep{dbp} packages for
%efficient translation of R-oriented data manipulations to 
%BigQuery SQL.  Throughput with HDF Cloud 
%is dependent upon the configuration of the object server,
%the relationship of numerical data layout to prevalent access
%patterns, and the degree to which queries capitalize on
%API efficiencies like chunk-based retrieval.  For both
%back ends, proper design and deployment of the querying client can
%lead to throughput that scale with client-side resources.

\section*{Conclusions}

Cloud-scale storage and retrieval strategies are of significant
interest for genome science.  
%The \Rclass{SummarizedExperiment} class
%unifies assay data with substantive sample- and experiment-level
%metadata, and its API for managing and interrogating
%genome-scale experiment archives is used in numerous
%analytic packages.  
The \Biocpackage{restfulSE} package exposes high-performance
cloud-resident data stores to users and
algorithms as \texttt{SummarizedExperiment}+s.  Continued improvements
in efficiency of
representation and query resolution for assay data and metadata
will help to achieve the potential of a federated data ecosystem for
enhanced discovery in biology through interactive genome-scale analysis.
Additional details and performance considerations are reviewed
in the Online Supplement to this paper.

\section*{Acknowledgments}
Support for the development of this software was provided by NIH grants
U01 CA214846 (Carey, PI) and U24 CA180996 (Morgan, PI).

\bibliography{BioC}

\end{document}
